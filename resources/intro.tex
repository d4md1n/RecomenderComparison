\section{Intro}
\paragraph{} The last decade Internet has been flooded with information. Information that no one can filter to find what he needs, raw data, videos, music or products. Large retail sites like amazon, developed recommenders systems in order to offer products to their users. The need although is not limited only in the retail area. Web sites like youtube or vimeo needs to recommend to each user of their, videos that may like to watch next. Facebook is another example of an application utilizing lots of data and offering recommendations on what you may want to read or who may be a friend of yours. Most of the times, a recommender system is not the core business functionality of an application. It is though a very useful feature that gives a clear advantage on any business area needed.

\paragraph{}The way a recommender system has been built is very dependent on the business case which will be served. Even a specific case of recommendation, similar to one already existing, might need a different recommender system.

\paragraph{} So far recommender systems has been very interesting area of study. Netflix in 2009 declared a challenge, which can be found here \footnote{http://www.netflixprize.com/}. Its  reward was one million dollars for the task of improving the accuracy of predictions. The prize was granted to BellKor’s Pragmatic Chaos team for their algorithm. You can come across this challenge on lots of papers published every year.

\paragraph{} With such a wide study of recommender systems, it is reasonable to start wondering "How are we going to compare recommender systems?". As we will discuss bellow, there are several papers suggesting ways of comparison. The majority of those papers are using the dataset given in the challenge above.

\paragraph{} In this thesis the first system to compare is a content-based recommendation system that provide predictions based on movies genre attributes. The second system is the based on the Alternating Least Squares (ALS) implementation of Apache Spark's MLlib.

\paragraph{} The comparison metrics used are the Mean Absolute Error (MAE), the Root Mean Squared Error and the ration between them (MAE/RMSE). Last but not least is the execution of time metric, measuring the training and estimation time.

\section{Related Work}
\paragraph{} In this thesis's section we will list numerous different approaches made in order to compare recommender systems.

\subsection{RecBench: Benchmarks for Evaluating Performance of Recommender System Architectures \cite{levandoski2011recbench}}
\paragraph{}University of Minnesota, published in 2011 a paper stating a comparison between a recommender framework and a DBMS-based recommender. In that paper they used the Movie Lens dataset 100k, from the Netflix Challenge. The benchmark had five areas of comparison. Those areas were initialization, pure recommendation, filtered recommendation, blended recommendation, item recommendation and item update.

\paragraph{}The initialization task was about the preparation needed for the system to go live. The next area was pure recommendation. By pure recommendation the author mean the home page recommendation, meaning the items that are going to be in the home page. Moving forward, we find the filtered recommendation. This recommendation is constrained by variables specific to the item, like movie genre etc. Another area of this evaluation contains the blended recommendation. Those recommendations are based on free text provided by the use in order to search. Item prediction is another are of the evaluation, in this prediction the user is navigated to the items page and the system is trying to predict the user's rating on the item. Last but not least, the paper examines the case of a new item being added to the system and how this is going to be incorporated to it.

\paragraph{}As a result, of those experiments the paper conclude that "hand-build recommenders exhibit superior performance in model building and pure recommendation tasks, while DBMS-based recommenders are superior to more complex recommendations such as providing filtered recommendations and blending text-search with recommendation prediction issues.

\subsection{Recommender Systems Evaluation: A 3D Benchmark \cite{said2012recommender}}
\paragraph{}In this paper the authors recognize the need for a common benchmark formula for recommender systems. This need lead them to propose one. They named it the 3D recommendation evaluation because they evaluate a system in three axis. These axis are business models, user requirements, and technical constrains. In business model axis they state that a recommender system must be evaluated on how well it serves the business case it is used for. In their paper they give the example of a video on demand service and evaluate it versus the pay per view business model and pay per subscription.

\paragraph{}In the user requirements axis, the evaluate the system based on what need it covers for the users. Is it, for example, going to reduce search time or decision making time.

\paragraph{}Last but not least is the technical constrains axis. In this axis the system is being evaluated based on data or hardware constrains, scalability and robustness.

\subsection{RiVal: A New Benchmarking Toolkit for Recommender Systems \cite{said2014rival}}
\paragraph{}RiVal, is an open source tool kit implemented in Java programming language. Rival is available via maven repositories. It is used in order to measure the evaluate recommender systems. Its evaluation is based on three points. Those point are data spiting, item recommendation, candidate item generation and performance measurement. 