\documentclass[12pt, a4paper, oneside, greek]{report}
\usepackage{url}
\usepackage{graphicx}
\usepackage{csvsimple}
\usepackage{pgfplots}
\usepackage{pgfplotstable}
\usepgfplotslibrary{groupplots}
\usepackage{amsmath}
\usepackage{algorithm}
\usepackage[noend]{algpseudocode}
\usepackage{listings}
\usepackage{setspace}
\usepackage[utf8]{inputenc}
\usepackage{alphabeta}
\usepackage[LGR,T1]{fontenc}

\onehalfspacing
\begin{document}
	\pagenumbering{gobble}
	\begin{titlepage}
		\centering
		\includegraphics[width=0.15\textwidth]{../images/UniversityOfPiraeusLogo.png}\par\vspace{1cm}
		{\scshape\LARGE University of Piraeus \par}
		\vspace{1cm}
		{\scshape\Large Department Of Digital Systems \par}
		\vspace{1cm}
		{\scshape\Large Post Graduate Program On \par Digital Systems and Services \par}
		\vspace{1cm}
		{\scshape\Large Master Thesis\par}
		\vspace{1cm}
		{\huge\bfseries Recommender Systems Comparison\par}
		\vspace{2cm}
		{\bfseries \large Σύγκριση Συστημάτων Προτάσεων\par}
		\vfill
		{\Large\itshape Vasileios Simeonidis\par}
	\end{titlepage}
	 accepted
	

Shamelessly pinched from this site, the following provides a Scala language definition for listings. Simply place it in your preamble, and use listings in the normal way.

% "define" Scala
\lstdefinelanguage{scala}{
  morekeywords={abstract,case,catch,class,def,%
    do,else,extends,false,final,finally,%
    for,if,implicit,import,match,mixin,%
    new,null,object,override,package,%
    private,protected,requires,return,sealed,%
    super,this,throw,trait,true,try,%
    type,val,var,while,with,yield},
  otherkeywords={=>,<-,<\%,<:,>:,\#,@},
  sensitive=true,
  morecomment=[l]{//},
  morecomment=[n]{/*}{*/},
  morestring=[b]",
  morestring=[b]',
  morestring=[b]"""
}


	\newpage
	\begin{center}
		This page was intentionally left blank.
	\end{center}
	\newpage 
	{
		\centering
		This thesis was supervised by\par
		\textit{Dr. Dimosthenis Kyriazis}
		\vfill
		{\large August 15, 2017\par}
	}
	\clearpage
	\begin{center}
		This page was intentionally left blank.
	\end{center}
	\newpage
	\chapter*{Acknowledgements}
	
	\paragraph{} This thesis couldn't be completed without the great support I received from so many people over this year. I would like to thank the following people.
	
	\paragraph{} My supervisor Dr. Dimosthenis Kyriazis for totally supporting me in the choices I made and giving me the freedom I was needing.
	
	\paragraph{} My friend and fellow student Dimitris Poulopoulos for helping me understand the field of recommender systems and supported me technically and theoretically.
	
	\paragraph{} My friends and colleagues George Adamopoulos and Nikos Silvestros for constantly teaching me high-level engineering and scientific thinking. Also, I would like to thank them for putting the right amount of pressure on me in order to complete this thesis.
	
	\paragraph{} Kronos, the development team I am part of and helping me to keep my spirit high. Thank you, Apostolos Chissas, Kostas Rigas, Christos Grivas, Peter Lengos, Vasiliki Giamarelou, Maria Karkeli, Eleni Karakizi, Spyros Argyroiliopoulos, Nikos Anagnostou and Ioannis Koutsileos. 
	
	\paragraph{} Last but not least, I would like to thank my close friends and family for bearing me while I was anxious about the completion of this thesis.
	
	\newpage
	\begin{center}
		This page was intentionally left blank.
	\end{center}
	\newpage
	\chapter*{Preface}
	\paragraph{} The last decade Internet has been flooded with information. Information that no one can filter to find what he needs, raw data, videos, music or products. Large retail sites like Amazon developed recommenders systems in order to offer products to their users. The need although is not limited only in the retail area. 
	\paragraph{}Web sites like Youtube or Vimeo need to recommend to each user of their, videos that may like to watch next. Facebook is another example of an application utilizing lots of data and offering recommendations on what you may want to read or who may be a friend of yours. Most of the times, a recommender system is not the core functionality of an application. It is through a very useful feature that gives a clear advantage in any business area needed.
	
	\paragraph{} This thesis aims to destiguise metrics on recommender systems  that can be proved useful to compare them. Also, this thesis, performs a comparison between two algorithms of the collaborative filtering family. The content based with focus on items and the machine learning oriented alternating least square (als).
	\newpage
	\chapter*{Πρόλογος}
	\paragraph{} Την τελευταία δεκαετία το διαδίκτυο έχει πλημμυρίσει από πληροφορία. Πληροφορία την οποία δεν μπορεί κάποιος να διαχειριστεί Αυτή η πληροφορία μπορεί να περιέχει από απλά δεδομένα μέχρι βίντεο, μουσική ή προϊόντα Μεγάλα καταστήματα όπως το Amazon ανάπτυξαν συστήματα προτάσεων για προτείνουν προϊόντα στους πελάτες τους πιο αποτελεσματικά Βέβαια η ανάγκη για συστήματα προτάσεων δεν περιορίζεται μόνο στον χώρο των πωλήσεων.
	
	\paragraph{} Ιστότοποι όπως το Youtube ή το Vimeo χρειάζονται να προτείνουν στους χρήστες τους βίντεο που μπορεί να τους αρέσει να παρακολουθήσουν στη συνέχεια. Η πλατφόρμα Facebook είναι ένα άλλο παράδειγμα για εφαρμογής που χρησιμοποιεί μεγάλο όγκο δεδομένων έτσι ώστε να είναι σε θέση να προτείνει τι μπορεί να θέλεις να διαβάσεις στη συνέχεια ή ποιος μπορεί να είναι φίλος σου. Τις περισσότερες φορές ένα σύστημα προτάσεων δεν είναι η κύρια λειτουργικότητα μίας εφαρμογής, είναι όμως ένα χαρακτηριστικό που μπορεί να δώσει ένα καθαρό προβάδισμα στην επιχειρησιακή περιοχή που χρειάζεται
	
	\paragraph{} Η διπλωματική αυτή εργασία στοχεύει στην ανάδειξη μετρικών πάνω σε συστήματα προτάσεων που μπορούν να φανούν χρήσιμα κατά την σύγκριση τους. Επίσης, το παρών έγγραφο, παρέχει μια σύγκριση μεταξύ δύο αλγορίθμων της οικογένειας collaborative filtering. Οι αλγόριθμοι αυτοί είναι ο content based με έμφαση στα αντικείμενα (items) και ο προσανατολισμένος στο machine learning, alternating least square (als).
	\newpage
	
	\tableofcontents
	\newpage
	\begin{center}
		This page was intentionally left blank.
	\end{center}
	\newpage
	\listoftables
	\newpage
	\begin{center}
		This page was intentionally left blank.
	\end{center}
	\newpage
	\listoffigures
	\newpage
	
	\begin{center}
		This page was intentionally left blank.
	\end{center}
	\newpage
	\pagenumbering{arabic}
	\part{Master Thesis}
	\section{Intro}
\paragraph{} This is the intro suction for this master thesis.
Why we need recommendation systems? Retailers can propose the right product to the right target group.
User get advertisements the may be interested in.\cite{RecommenderSystems:2}

\paragraph{}
History, what has been tried so far?

\section{Related Work}
	\section{Collaborative filtering}
\subsection{Content based}
\subsection{Latent Factors}
test
sadsad

sad

asd

ds
\begin{figure}[ht]
\centering
\includegraphics[width=0.7\linewidth]{images/equation01}
\caption{\bfseries Equation}
\label{fig:equation02}
\end{figure}
	\section{Our Experiment}
\subsection{Infrastructure}
\subsubsection{Apache Spark}
What is map reduce\\
How spark differentiates from its predecessors, hadoop yarn\\
Resilient Distributed Datasets (RDDs) \\
mllibs\\
add spark jira note \\
broadcasting rdd \\
//cite the mastering apache spark book
\cite{ApacheSpark:1} \\
a Spark cluster to be created on AWS EC2 storage.\\
New trends on spark https://github.com/apache-spark-on-k8s/spark cite this repository too.
\subsection{Dataset}
What is the dataset about. This dataset contains users, movies and the rating user made about the movies.
This dataset is splited to multiple subsets of 80000 training sets and respective 20000 reviews.
\cite{MovieLens:3}
\subsection{Metrics}
\subsubsection{Mean Absolute Error}
\begin{equation}
MAE = \frac{\sum_{i=1}^{n}{|y_{i}-x_{i}|} }{n} = \frac{\sum_{i=1}^{n}\sqrt{{(y_{i}-x_{i})}^{2}}}{n}
\end{equation}
\subsubsection{Execution Time}
Time is measured in milliseconds.
	\section{Results}
\begin{table}[ht]
		\caption {\bfseries Content Based Algorithm Results}
\begin{tabular}{l|l|r|r}%
   	\bfseries Training Dataset & \bfseries Testing Dataset & \bfseries Mean Absolute Error & \bfseries  Execution time (ms)% specify table head
   	\csvreader[head to column names]{data/contentBased.csv}{}% use head of csv as column names
   	{\\\hline \trainingSet & \testingSet & \MAE & \ExecutionTime}% specify your coloumns here
\end{tabular}
  \label{tab:Content Based Algorithm Results}
\end{table}

\pgfplotstableread[col sep=comma]{data/contentBased.csv}\latentFactorsDataTable
\begin{tikzpicture}
\begin{axis}[
title={Content Based - Mean Absolute Error},
xlabel= Dataset,
ylabel=Mean Absolute Error,
width=1\linewidth, 
xtick=data,
xticklabels from table={\latentFactorsDataTable}{trainingSet}]
\addplot [ybar, fill=blue] table [x expr=\coordindex, y=MAE, col sep=comma] {data/contentBased.csv};
\end{axis}
\end{tikzpicture} \\ \\
\begin{tikzpicture}
\begin{axis}[
title={Content Based - Execution Time},
xlabel= Dataset,
ylabel=Execution Time,
width=1\linewidth, 
xtick=data,
xticklabels from table={\latentFactorsDataTable}{trainingSet}]
\addplot [ybar, fill=blue] table [x expr=\coordindex, y=ExecutionTime, col sep=comma] {data/contentBased.csv};
\end{axis}
\end{tikzpicture}


\begin{table}[ht]
		\caption{\bfseries Latent Factors Algorithm Results}
\begin{tabular}{l|l|r|r}%
	\bfseries Training Dataset & \bfseries Testing Dataset & \bfseries Mean Absolute Error & \bfseries  Execution time (ms)% specify table head
	\csvreader[head to column names]{data/latentFactors.csv}{}% use head of csv as column names
	{\\\hline \trainingSet & \testingSet & \MAE & \ExecutionTime}% specify your coloumns here
\end{tabular}
  \label{tab:Latent Factors Algorithm Results}
\end{table}

\pgfplotstableread[col sep=comma]{data/latentFactors.csv}\latentFactorsDataTable
\begin{tikzpicture}
\begin{axis}[
title={Latent Factors - Mean Absolute Error},
xlabel= Dataset,
ylabel=Mean Absolute Error,
width=1\linewidth, 
xtick=data,
xticklabels from table={\latentFactorsDataTable}{trainingSet}]
\addplot [ybar, fill=blue] table [x expr=\coordindex, y=MAE, col sep=comma] {data/latentFactors.csv};
\end{axis}
\end{tikzpicture} \\ \\
\begin{tikzpicture}
\begin{axis}[
title={Latent Factors - Execution Time},
xlabel= Dataset,
ylabel=Execution Time,
width=1\linewidth, 
xtick=data,
xticklabels from table={\latentFactorsDataTable}{trainingSet}]
\addplot [ybar, fill=blue] table [x expr=\coordindex, y=ExecutionTime, col sep=comma] {data/latentFactors.csv};
\end{axis}
\end{tikzpicture}
	\newpage
\section{Conclusion}
As a conclusion we can see that als is better on both metrics from the content based.
	\newpage
	\bibliography{thesis}
	\bibliographystyle{ieeetr}
	\newpage
\appendix
\part{Appendices}
In this part you will find the code used for this master thesis.
\section{Code used}
Below is the code used in order to produce the results. It has been 
written for Apache Spark \cite{ApacheSpark:1} using scala. The library breeze has been used for matrix processing.
\subsection{Item Based Collaborative Filtering}
\lstinputlisting[language=scala]{../src/main/scala/ContentBased.scala}
\subsection{Latent Factors}
\lstinputlisting[language=scala]{../src/main/scala/LatentFactors.scala}
\subsection{Infrastructure code}
\lstinputlisting[language=scala]{../src/main/scala/Infrastructure.scala}
\lstinputlisting[language=scala]{../src/main/scala/Metrics.scala}
\listoftables
\newpage
\listoffigures
\end{document}