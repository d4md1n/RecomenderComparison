\documentclass[10pt,journal]{IEEEtran}
\usepackage{url}
\usepackage{graphicx}
\usepackage{csvsimple}
\usepackage{pgfplots}
\usepackage{pgfplotstable}
\usepgfplotslibrary{groupplots}
\usepackage{amsmath}
\usepackage{algorithm}
\usepackage[noend]{algpseudocode}
\usepackage{listings}
\usepackage{setspace}
\usepackage[utf8]{inputenc}

\begin{document}
	
	\title{Recommender Systems} % to be concidered
	\author{
		\IEEEauthorblockN{Vasileios Simeonidis}
		\IEEEauthorblockA{\\School of Information \\
			and Communication Technologies\\
			Department of Digital Systems\\
			University of Piraeus\\
			Email: vsimeonidis@outlook.com}
		}
	\maketitle
	\begin{abstract}
		The last decade Internet has been flooded with information. Information that no one can filter to find what he needs, raw data, videos, music or products. Large retail sites like Amazon developed recommenders systems in order to offer products to their users. The need although is not limited only in the retail area. 
		
		Web sites like Youtube or Vimeo need to recommend to each user of their, videos that may like to watch next. Facebook is another example of an application utilizing lots of data and offering recommendations on what you may want to read or who may be a friend of yours. Most of the times, a recommender system is not the core functionality of an application. It is through a very useful feature that gives a clear advantage in any business area needed.
		
		This thesis aims to destiguise metrics on recommender systems  that can be proved useful to compare them. Also, this thesis, performs a comparison between two algorithms of the collaborative filtering family. The content based with focus on items and the machine learning oriented alternating least square (als).
	\end{abstract}
	\begin{IEEEkeywords}
		IEEEtran, journal, \LaTeX, paper, template.
	\end{IEEEkeywords}

	
%	\tableofcontents
%	\listoftables
%	\newpage
%	\listoffigures
	\pagenumbering{arabic}
	\section{Intro}
\paragraph{} This is the intro suction for this master thesis.
Why we need recommendation systems? Retailers can propose the right product to the right target group.
User get advertisements the may be interested in.\cite{RecommenderSystems:2}

\paragraph{}
History, what has been tried so far?

\section{Related Work}
	\section{Collaborative filtering}
\subsection{Content based}
\subsection{Latent Factors}
test
sadsad

sad

asd

ds
\begin{figure}[ht]
\centering
\includegraphics[width=0.7\linewidth]{images/equation01}
\caption{\bfseries Equation}
\label{fig:equation02}
\end{figure}
	\section{Our Experiment}
\subsection{Infrastructure}
\subsubsection{Apache Spark}
What is map reduce\\
How spark differentiates from its predecessors, hadoop yarn\\
Resilient Distributed Datasets (RDDs) \\
mllibs\\
add spark jira note \\
broadcasting rdd \\
//cite the mastering apache spark book
\cite{ApacheSpark:1} \\
a Spark cluster to be created on AWS EC2 storage.\\
New trends on spark https://github.com/apache-spark-on-k8s/spark cite this repository too.
\subsection{Dataset}
What is the dataset about. This dataset contains users, movies and the rating user made about the movies.
This dataset is splited to multiple subsets of 80000 training sets and respective 20000 reviews.
\cite{MovieLens:3}
\subsection{Metrics}
\subsubsection{Mean Absolute Error}
\begin{equation}
MAE = \frac{\sum_{i=1}^{n}{|y_{i}-x_{i}|} }{n} = \frac{\sum_{i=1}^{n}\sqrt{{(y_{i}-x_{i})}^{2}}}{n}
\end{equation}
\subsubsection{Execution Time}
Time is measured in milliseconds.
	\section{Results}
\begin{table}[ht]
		\caption {\bfseries Content Based Algorithm Results}
\begin{tabular}{l|l|r|r}%
   	\bfseries Training Dataset & \bfseries Testing Dataset & \bfseries Mean Absolute Error & \bfseries  Execution time (ms)% specify table head
   	\csvreader[head to column names]{data/contentBased.csv}{}% use head of csv as column names
   	{\\\hline \trainingSet & \testingSet & \MAE & \ExecutionTime}% specify your coloumns here
\end{tabular}
  \label{tab:Content Based Algorithm Results}
\end{table}

\pgfplotstableread[col sep=comma]{data/contentBased.csv}\latentFactorsDataTable
\begin{tikzpicture}
\begin{axis}[
title={Content Based - Mean Absolute Error},
xlabel= Dataset,
ylabel=Mean Absolute Error,
width=1\linewidth, 
xtick=data,
xticklabels from table={\latentFactorsDataTable}{trainingSet}]
\addplot [ybar, fill=blue] table [x expr=\coordindex, y=MAE, col sep=comma] {data/contentBased.csv};
\end{axis}
\end{tikzpicture} \\ \\
\begin{tikzpicture}
\begin{axis}[
title={Content Based - Execution Time},
xlabel= Dataset,
ylabel=Execution Time,
width=1\linewidth, 
xtick=data,
xticklabels from table={\latentFactorsDataTable}{trainingSet}]
\addplot [ybar, fill=blue] table [x expr=\coordindex, y=ExecutionTime, col sep=comma] {data/contentBased.csv};
\end{axis}
\end{tikzpicture}


\begin{table}[ht]
		\caption{\bfseries Latent Factors Algorithm Results}
\begin{tabular}{l|l|r|r}%
	\bfseries Training Dataset & \bfseries Testing Dataset & \bfseries Mean Absolute Error & \bfseries  Execution time (ms)% specify table head
	\csvreader[head to column names]{data/latentFactors.csv}{}% use head of csv as column names
	{\\\hline \trainingSet & \testingSet & \MAE & \ExecutionTime}% specify your coloumns here
\end{tabular}
  \label{tab:Latent Factors Algorithm Results}
\end{table}

\pgfplotstableread[col sep=comma]{data/latentFactors.csv}\latentFactorsDataTable
\begin{tikzpicture}
\begin{axis}[
title={Latent Factors - Mean Absolute Error},
xlabel= Dataset,
ylabel=Mean Absolute Error,
width=1\linewidth, 
xtick=data,
xticklabels from table={\latentFactorsDataTable}{trainingSet}]
\addplot [ybar, fill=blue] table [x expr=\coordindex, y=MAE, col sep=comma] {data/latentFactors.csv};
\end{axis}
\end{tikzpicture} \\ \\
\begin{tikzpicture}
\begin{axis}[
title={Latent Factors - Execution Time},
xlabel= Dataset,
ylabel=Execution Time,
width=1\linewidth, 
xtick=data,
xticklabels from table={\latentFactorsDataTable}{trainingSet}]
\addplot [ybar, fill=blue] table [x expr=\coordindex, y=ExecutionTime, col sep=comma] {data/latentFactors.csv};
\end{axis}
\end{tikzpicture}
	\newpage
\section{Conclusion}
As a conclusion we can see that als is better on both metrics from the content based.

	\section{Acknowledgements}
	
	\paragraph{} This work couldn't be completed without the great support I received from so many people over this year. I would like to thank the following people.
	
	\paragraph{} My supervisor Dr. Dimosthenis Kyriazis for totally supporting me in the choices I made and giving me the freedom I was needing.
	
	\paragraph{} My friend and fellow student Dimitris Poulopoulos for helping me understand the field of recommender systems and supported me technically and theoretically.
	
	\paragraph{} My friends and colleagues George Adamopoulos and Nikos Silvestros for constantly teaching me high-level engineering and scientific thinking. Also, I would like to thank them for putting the right amount of pressure on me in order to complete this thesis.
	
	\paragraph{} Kronos, the development team I am part of and helping me to keep my spirit high. Thank you, Apostolos Chissas, Kostas Rigas, Christos Grivas, Peter Lengos, Vasiliki Giamarelou, Maria Karkeli, Eleni Karakizi, Spyros Argyroiliopoulos, Nikos Anagnostou and Ioannis Koutsileos. 
	
	\paragraph{} Last but not least, I would like to thank my close friends and family for bearing me while I was anxious about the completion of this thesis.
	
	\bibliography{paper}
	\bibliographystyle{ieeetr}
\end{document}