\section{Results}

\paragraph{} In this chapter, we will examine the results taken from the experiment described in the previous chapter. 
In order to measure the performance of each recommender system, we used three different metrics. Those metrics are the Mean Absolute Error (MAE), the Root Mean Square Error (RMSE), and the execution time of each system.

\paragraph{} The Mean Absolute Error is defined as the sum of each error's absolute value divided by the number of them. Let's take a look
at error's definition. It is common in statistics to symbolize the error with $e_i$. The $i$ index shows which measurement's error is. An error can be calculated using the following formula $e_i = \widehat{y_i} - y_i$, referring to $\widehat{y_{i}}$ as the estimated value for the i-th item whereas to $y_i$ as the actual value of the i-th item. For example if we estimated that the $\widehat{y_i} = 5$ and its actual value is $y_i = 5.2$ and its error could be the following.
\begin{equation}
e_i = \widehat{y_i} - y_i => e_i = 5 - 5.2  => e_i = -0.2
\end{equation}

\paragraph{} Mean absolute error is easier to interpret than other statistical metrics like RMSE. We will examine RMSE later in this chapter. If we want to express mathematically the MAE we could write the following.

\begin{equation}
MAE = \frac{\sum_{i=1}^{n}{|e_i|}}{n} =
\frac{\sum_{i=1}^{n}{|\widehat{y_{i}}-y_{i}|} }{n} =
\frac{\sum_{i=1}^{n}\sqrt{{(\widehat{y_{i}}-y_{i})}^{2}}}{n}
\end{equation}

\paragraph{}During the experiment we took the MAE metric for each system. The two following tables can show the results in detail.
\begin{table}[htb]
	\caption {\bfseries Content Based Algorithm Results vs Mean Absolute Error}
	\centering
	\begin{tabular}{c|r}%
		\bfseries Training Dataset \ | Testing Dataset & \bfseries Mean Absolute Error
		\csvreader[head to column names]{../data/contentBased.csv}{}% use head of csv as column names
		{\\\hline \trainingSet \ | \testingSet & \MAE}% specify your columns here
	\end{tabular}
	\label{tab:Content Based Algorithm Results vs MAE}
\end{table}
\begin{table}[htb]
	\caption{\bfseries Latent Factors Algorithm Results vs Mean Absolute Error}
	\centering
	\begin{tabular}{c|r}%
		\bfseries Training Dataset \ | Testing Dataset & \bfseries Mean Absolute Error
		\csvreader[head to column names]{../data/latentFactors.csv}{}% use head of csv as column names
		{\\\hline \trainingSet \ | \testingSet & \MAE}% specify your columns here
	\end{tabular}
	\label{tab:Latent Factors Algorithm Results vs MAE}
\end{table}
\paragraph{}As we can see, comparing the two systems on this metric we realize that the latent factors system outperformed the content based one. That is clearly depicted on the graph below.
\pgfplotstableread[col sep=comma]{../data/contentBased.csv}\contentBasedDataTable
\pgfplotstableread[col sep=comma]{../data/latentFactors.csv}\latentFactorsDataTable
\begin{figure}[ht]
	\centering
	\begin{tikzpicture}
	\begin{axis}[
	xlabel= Dataset,
	ylabel=Mean Absolute Error,
	width=1\linewidth, 
	xtick=data,
	xticklabels from table={\latentFactorsDataTable}{trainingSet},
	ymax = 1.8
	]
	\addplot [ybar, fill=red, bar shift=-.3cm, area legend] table [x expr=\coordindex, y=MAE, col sep=comma] {\contentBasedDataTable};
	\addplot [ybar, fill=blue, bar shift=.3cm, area legend] table [x expr=\coordindex, y=MAE, col sep=comma] {\latentFactorsDataTable};
	\addlegendentry{Content Based}
	\addlegendentry{Latent Factors}
	\end{axis}
	\end{tikzpicture}
	\caption{\bfseries Latent Factors vs Content Based on Mean Absolute Error}\label{MAE_Comparison}
\end{figure}

\paragraph{} In bibliography the most common statistical metric for recommender systems is the root mean square error(RMSE). This metric is considered to give a greater estimation of error magnitude due to the fact that it uses the squared value of each error. In order to better understand this metric, we can have a look at its mathematical type below.

\begin{equation}
RMSE = \sqrt{\frac{\sum_{i=1}^{n}{e_i^2}}{n}} =
\sqrt{\frac{\sum_{i=1}^{n}{(\widehat{y_{i}}-y_{i})^2} }{n}}
\end{equation}

\paragraph{} Each error is measured as before but now due to the structure of that metric the larger the error it is, the greatest the impact it has on RMSE.

\paragraph{}The results we got during the experiment on that metric was that latent factors system outperformed the content base again. You can see the details of the result on the two following tables.

\begin{table}[ht]
	\caption {\bfseries Content Based Algorithm Results vs Root Mean Square Error}
	\centering
	\begin{tabular}{c|r}%
		\bfseries Training Dataset \ | Testing Dataset & \bfseries Root Mean Square Error
		\csvreader[head to column names]{../data/contentBased.csv}{}% use head of csv as column names
		{\\\hline \trainingSet \ | \testingSet & \RMSE}% specify your columns here
	\end{tabular}
	\label{tab:Content Based Algorithm Results vs RMSE}
\end{table}

\begin{table}[ht]
	\caption{\bfseries Latent Factors Algorithm Results vs Root Mean Square Error}
	\centering
	\begin{tabular}{c|r}%
		\bfseries Training Dataset \ | Testing Dataset & \bfseries Root Mean Square Error
		\csvreader[head to column names]{../data/latentFactors.csv}{}% use head of csv as column names
		{\\\hline \trainingSet \ | \testingSet & \RMSE}% specify your columns here
	\end{tabular}
	\label{tab:Latent Factors Algorithm Results vs RMSE}
\end{table}
The results above can be shown clearly on the graph following, depicting the performance of latent factors and content based systems, on each dataset, on RMSE metric.
\begin{figure}[ht]
	\centering
	\begin{tikzpicture}
	\begin{axis}[
	xlabel= Dataset,
	ylabel=Root Mean Square Error,
	width=1\linewidth, 
	xtick=data,
	xticklabels from table={\latentFactorsDataTable}{trainingSet},
	ymax = 2.3
	]
	\addplot [ybar, fill=red, bar shift=-.3cm, area legend] table [x expr=\coordindex, y=RMSE, col sep=comma] {\contentBasedDataTable};
	\addplot [ybar, fill=blue, bar shift=.3cm, area legend] table [x expr=\coordindex, y=RMSE, col sep=comma] {\latentFactorsDataTable};
	\addlegendentry{Content Based}
	\addlegendentry{Latent Factors}
	\end{axis}
	\end{tikzpicture}
	\caption{\bfseries Latent Factors vs Content Based on Root Mean Squeare Error}\label{RMSE_Comparison}
\end{figure}

\paragraph{} Even if RMSE is considered a better metric when large errors are undesired, it is useful to check those systems on the $\frac{MAE}{RMSE}$ metric too. It is easily proven that $MAE \leq RMSE$. Those two are equal when the magnitude of all errors is the same. Examining this ratio we can see if the magnitude of the errors has close values.
\begin{table}[ht]
	\caption {\bfseries Content Based Algorithm Results vs MAE over RMSE}
	\centering
	\begin{tabular}{c|r}%
		\bfseries Training Dataset \ | Testing Dataset & \bfseries  MAE\  \textbackslash \ RMSE
		\csvreader[head to column names]{../data/contentBased.csv}{}% use head of csv as column names
		{\\\hline \trainingSet \ | \testingSet & \MAEoverRMSE}% specify your columns here
	\end{tabular}
	\label{tab:Content Based Algorithm Results vs MAE over RMSE}
\end{table}

\begin{table}[ht]
	\caption{\bfseries Latent Factors Algorithm Results vs MAE over RMSE}
	\centering
	\begin{tabular}{c|r}%
		\bfseries Training Dataset \ | Testing Dataset 
		& \bfseries  MAE\  \textbackslash \ RMSE
		\csvreader[head to column names]{../data/latentFactors.csv}{}% use head of csv as column names
		{\\\hline \trainingSet \ | \testingSet & \MAEoverRMSE}% specify your columns here
	\end{tabular}
	\label{tab:Latent Factors Algorithm Results vs MAE over RMSE}
\end{table}

\paragraph{}As it was expected and this area of examination, latent factors system has ten percent (10\%) fewer error spikes than the content based on every dataset. The graph below depicts the results.
\begin{figure}[ht]
	\centering
	\begin{tikzpicture}
	\begin{axis}[
	xlabel= Dataset,
	ylabel= MAE \ \textbackslash \ RMSE,
	width=1\linewidth, 
	xtick=data,
	xticklabels from table={\latentFactorsDataTable}{trainingSet},
	ymax = 1
	]
	\addplot [ybar, fill=red, bar shift=-.3cm, area legend] table [x expr=\coordindex, y=MAEoverRMSE, col sep=comma] {\contentBasedDataTable};
	\addplot [ybar, fill=blue, bar shift=.3cm, area legend] table [x expr=\coordindex, y=MAEoverRMSE, col sep=comma] {\latentFactorsDataTable};
	\addlegendentry{Content Based}
	\addlegendentry{Latent Factors}
	\end{axis}
	\end{tikzpicture}
	\caption{\bfseries Latent Factors vs Content Based on MAE over RMSE}\label{MAEoverRMSE_Comparison}
\end{figure}

\paragraph{} The last metric we took in order to compare those two systems was the execution time. On execution time metric we included the time need to train the system against a data set and the time needed to calculate the metrics. We extracted the methods used on metrics calculation in order to be commonly used and impact each execution time result on the same level. The results can be found in the tables below and on the graph that visualizes them, also below.


\begin{table}[ht]
	\caption {\bfseries Content Based Algorithm Results vs Execution Time}
	\centering
	\begin{tabular}{c|r}%
		\bfseries Training Dataset \ | Testing Dataset & \bfseries  Execution Time (ms)% specify table head
		\csvreader[head to column names]{../data/contentBased.csv}{}% use head of csv as column names
		{\\\hline \trainingSet \ | \testingSet & \ExecutionTime}% specify your columns here
	\end{tabular}
	\label{tab:Content Based Algorithm Results vs Execution Time}
\end{table}

\begin{table}[ht]
	\caption{\bfseries Latent Factors Algorithm Results vs Execution Time}
	\centering
	\begin{tabular}{c|r}%
		\bfseries Training Dataset \ | Testing Dataset & \bfseries  Execution time (ms)% specify table head
		\csvreader[head to column names]{../data/latentFactors.csv}{}% use head of csv as column names
		{\\\hline \trainingSet \ | \testingSet & \ExecutionTime}% specify your columns here
	\end{tabular}
	\label{tab:Latent Factors Algorithm Results vs Execution Time}
\end{table}

\begin{figure}[ht]
	\centering
	\begin{tikzpicture}
	\begin{axis}[
	xlabel= Dataset,
	ylabel=Execution Time,
	width=1\linewidth, 
	xtick=data,
	xticklabels from table={\latentFactorsDataTable}{trainingSet},
	legend entries={entry1,entry2}]
	\addplot [ybar, fill=red, bar shift=-.3cm, area legend] table [x expr=\coordindex, y=ExecutionTime, col sep=comma] {\contentBasedDataTable};
	\addplot [ybar, fill=blue, bar shift=.3cm, area legend] table [x expr=\coordindex, y=ExecutionTime, col sep=comma] {\latentFactorsDataTable};
	\addlegendentry{Content Based}
	\addlegendentry{Latent Factors}
	\end{axis}
	\end{tikzpicture}
	\caption{\bfseries Latent Factors vs Content Based on Execution Time}\label{ET_Comparison}
\end{figure}

\paragraph{}Comparing those systems in the execution time metric we can also see that latent factors system outperformed the content base again by taking one-third of the time in its worst case.