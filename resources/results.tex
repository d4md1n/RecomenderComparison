\newpage
\section{Results}
Gathering the results of the above experiment, we can see roughly a very big difference between the two algorithms on both metrics, MAE and Execution Time.


%consider run the process for the 1M dataset
%Add dataset statistics and refernce the papers using it

\begin{table}[!h]
		\caption {\bfseries Content Based Algorithm Results}
\begin{tabular}{l|l|r|r}%
   	\bfseries Training Dataset & \bfseries Testing Dataset & \bfseries Mean Absolute Error & \bfseries  Execution time (ms)% specify table head
   	\csvreader[head to column names]{data/contentBased.csv}{}% use head of csv as column names
   	{\\\hline \trainingSet & \testingSet & \MAE & \ExecutionTime}% specify your columns here
\end{tabular}
  \label{tab:Content Based Algorithm Results}
\end{table}


\begin{table}[!h]
		\caption{\bfseries Latent Factors Algorithm Results}
\begin{tabular}{l|l|r|r}%
	\bfseries Training Dataset & \bfseries Testing Dataset & \bfseries Mean Absolute Error & \bfseries  Execution time (ms)% specify table head
	\csvreader[head to column names]{data/latentFactors.csv}{}% use head of csv as column names
	{\\\hline \trainingSet & \testingSet & \MAE & \ExecutionTime}% specify your columns here
\end{tabular}
  \label{tab:Latent Factors Algorithm Results}
\end{table}
\pgfplotstableread[col sep=comma]{data/contentBased.csv}\contentBasedDataTable
\pgfplotstableread[col sep=comma]{data/latentFactors.csv}\latentFactorsDataTable
\begin{figure}[!h]
\centering
\begin{tikzpicture}[scale=0.5]
\begin{axis}[
xlabel= Dataset,
ylabel=Mean Absolute Error,
width=1\linewidth, 
xtick=data,
xticklabels from table={\latentFactorsDataTable}{trainingSet},
ymax = 1.8
]
\addplot [ybar, fill=red, bar shift=-.3cm, area legend] table [x expr=\coordindex, y=MAE, col sep=comma] {\contentBasedDataTable};
\addplot [ybar, fill=blue, bar shift=.3cm, area legend] table [x expr=\coordindex, y=MAE, col sep=comma] {\latentFactorsDataTable};
\addlegendentry{Content Based}
\addlegendentry{Latent Factors}
\end{axis}
\end{tikzpicture}
\caption{\bfseries Latent Factors vs Content Based on Mean Absolute Value}\label{MAE_Comparison}
\end{figure}
\begin{figure}[!h]
\centering
\begin{tikzpicture}[scale=0.5]
\begin{axis}[
xlabel= Dataset,
ylabel=Execution Time,
width=1\linewidth, 
xtick=data,
xticklabels from table={\latentFactorsDataTable}{trainingSet},
legend entries={entry1,entry2}]
\addplot [ybar, fill=red, bar shift=-.3cm, area legend] table [x expr=\coordindex, y=ExecutionTime, col sep=comma] {\contentBasedDataTable};
\addplot [ybar, fill=blue, bar shift=.3cm, area legend] table [x expr=\coordindex, y=ExecutionTime, col sep=comma] {\latentFactorsDataTable};
\addlegendentry{Content Based}
\addlegendentry{Latent Factors}
\end{axis}
\end{tikzpicture}
\caption{\bfseries Latent Factors vs Content Based on Execution Time}\label{ET_Comparison}
\end{figure}